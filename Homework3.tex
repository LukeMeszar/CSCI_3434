\documentclass{article}
\usepackage[english]{babel}
\usepackage{amsmath}
\usepackage{amssymb}
\usepackage{amsthm}
\usepackage{amsfonts}
\usepackage{blindtext}
\usepackage{systeme}
\usepackage{relsize}
\usepackage{enumerate}
\usepackage{commath}
\usepackage{mathrsfs}
\usepackage{bm}
\usepackage{float}
\usepackage{graphicx}
\usepackage{wrapfig}
\usepackage{mathtools}
\usepackage{subcaption}
\usepackage{tikz-cd}
\usetikzlibrary{automata,positioning}
%\usepackage{hyperref}
%\hypersetup{
% colorlinks=false,% hyperlinks will be black
%linkbordercolor=black,% hyperlink borders will be red
%pdfborderstyle={/S/U/W 0.25}% border style will be underline of width 1pt
%}
\usepackage{chngcntr}
\usepackage{MnSymbol}
\usepackage{tikz}
\usepackage[margin=0.75in]{geometry}
\usepackage{cleveref}

\newcommand{\C}{\mathbf{C}}
\newcommand{\R}{\mathbf{R}}
\newcommand{\Q}{\mathbf{Q}}
\newcommand{\Z}{\mathbf{Z}}
\newcommand{\N}{\mathbf{N}}
\newcommand{\E}{\mathbf{E}}
\newcommand{\B}{\mathbf{B}}
\newcommand{\F}{\mathbf{F}}
\newcommand{\U}{\mathbf{U}}
\newcommand{\V}{\mathbf{V}}
\newcommand{\J}{\mathbf{J}}
\newcommand{\x}{\mathbf{x}}
\newcommand{\y}{\mathbf{y}}
\newcommand{\z}{\mathbf{z}}
\newcommand{\Sb}{\mathbf{S}}
\newcommand{\Cs}{\mathscr{C}}
\newcommand{\As}{\mathscr{A}}
\newcommand{\I}{\textnormal{\textbf{I}}}
\newcommand{\Id}{\dot{\textnormal{\textbf{I}}}}
\newcommand{\Top}{\textnormal{\textbf{Top}}}
\newcommand{\hTop}{\textnormal{\textbf{hTop}}}
\newcommand{\Groups}{\textnormal{\textbf{Groups}}}
\newcommand{\Set}{\textnormal{\textbf{Set}}}
\newcommand{\Xib}{\mathbf{\Xi}}
\newcommand{\Sym}{\text{Sym}}
\newcommand{\prid}[1]{\langle #1 \rangle}
\newcommand{\apoly}[1]{a_{#1} + a_{#1}x+ a_{#1}x^2 + \cdots + a_{#1}x^n}
\newcommand{\dist}{\textnormal{dist}}
\newcommand{\ti}[1]{\textit{#1}}
\newcommand{\tb}[1]{\textnormal{\textbf{#1}}}
\newcommand{\es}{\varnothing}
\newcommand{\sst}{\subset}
\newcommand{\ssteq}{\subseteq}
\newcommand{\func}[3]{#1: #2 \to #3}
\newcommand{\inte}[1]{\textnormal{int}(#1)}
\newcommand{\bdr}[1]{\textnormal{bdry}(#1)}
\newcommand{\ifff}{if and only if }
\newcommand{\st}{such that }
\newcommand{\wrt}{with respect to }
\newcommand{\tspace}[1]{\text{T}_#1}
\newcommand{\mathdash}{\hbox{-}}
\newcommand{\diam}[1]{\textnormal{diam}(#1)}
\newcommand{\setst}{\hspace{1mm} | \hspace{1mm} }
\newcommand{\supp}{\textnormal{support}}
\newcommand{\clos}{\textnormal{closure}}
\newcommand{\rel}{\textnormal{rel }}
\newcommand{\Hom}{\textnormal{Hom}}
\newcommand{\obj}{\textnormal{obj}}
\newcommand{\varlisto}[2]{#1_1,#1_2,\ldots,#1_{#2}}
\newcommand{\varlistz}[2]{#1_0,#1_1,\ldots,#1_{#2}}
\newcommand{\finv}[2]{#1^{-1}(#2)}
\newcommand{\disu}{\rotatebox[origin=c]{90}{$\models$}}
\newcommand{\rank}{\textnormal{rank }}
\newcommand{\card}{\textnormal{card }}
\newcommand{\im}{\textnormal{im }}
\newcommand{\cls}{\textnormal{cls }}
\newcommand{\rev}{\textnormal{rev }}
\newcommand{\defeq}{\mathrel{\stackrel{\makebox[0pt]{\mbox{\normalfont\tiny def}}}{=}}}



%\renewcommand{\phi}{\varphi}
\renewcommand{\epsilon}{\varepsilon}


\newtheorem{theorem}{Theorem}[section]
\newtheorem{corollary}[theorem]{Corollary}
\newtheorem{proposition}{Proposition}[theorem]
\newtheorem{lemma}[theorem]{Lemma}
\theoremstyle{definition}
\newtheorem*{definition}{Definition}
\newtheorem*{remark}{Remark}
\newtheorem*{remarks}{Remarks}


\renewcommand\qedsymbol{$\blacksquare$}
\setcounter{section}{-1}

\counterwithin*{equation}{section}

\title{CSCI 3434 Theory of Computation Homework 3}
\author{Luke Meszar (Worked with Jeff Lipnick)}
\date{October 18, 2017}
\begin{document}
\maketitle
\begin{enumerate}
	\item[HW 5.2]
	Prove that the CFG 
	\[S \rightarrow aSb \setst bSa \setst SS \setst \epsilon\]
	generates the set of all strings over $\{a,b\}$ with equally many $a$'s and $b$'s. (Hint: Characterize elements of the set in terms of the graph of
	the function $\#b(y)-\#a(y)$ as y ranges over prefixes of $x$, as we did
	in Lecture 20 with balanced parentheses.)
	\item[HW 6.1] Prove that the following CFG $G$ in Greibach normal form generates
	exactly the set of nonnull strings over $\{a, b\}$ with equally many $a$'s and
	$b$'s:
	\begin{align*}
	S &\rightarrow aB \setst bA, \\
	A &\rightarrow aS \setst bAA \setst a, \\
	B &\rightarrow bS \setst aBB \setst b.
	\end{align*}
	(Hint: Strengthen your induction hypothesis to describe the sets of
	strings generated by the nonterminals $A$ and $B$: for $x \neq \epsilon$,
	\begin{align*}
	S &\xrightarrow[G]{*} \iff \#a(x) = \#b(x),\\
	A &\xrightarrow[G]{*}\ ???,\\
	B &\xrightarrow[G]{*}\ ???.)\\
	\end{align*}
	\item[Misc. 1]
	Closure properties of CFLs. For each of the following, let $L_1$ and $L_2$ be arbitrary CFLs and determine whether the given language must be a CFL (YES) or can you find a counterexample (NO), i.e., particular choices of $L_1$ and $L_2$ such that the language is not a CFL. For the YES's give a short justification involving the construction of a grammar/PDA from the given grammars/PDAs. For t he NO's, use the CFL pumping lemma or known non-context-free languages to establish your counterexample.
	\begin{enumerate}
		\item $L_1L_2$
		\item $L_1 \cap L_2$
		\item $L_1 \cup L_2$ 
		\item $L_1^*$
	\end{enumerate} 
\item[Misc. 2] For each language, decide whether it is regular, not regular but context-free, or not context-free. Justify your answer. 
\begin{enumerate}
	\item $\{a^k b^l a^m b^n \setst k = m \text{ or } l = n \}$ 
	\item $\{a^k b^l a^m b^n \setst k = m \text{ and } l = n\}$
\end{enumerate}
\item[Misc. 3] Give a PDA accepting the following languages.  No proof required.
\begin{enumerate}
\item  $\{a^n b^m c^k \setst k = n + m\}$ 
\item $\{x \in \{a,b\}^* \setst \#a(x) = \#b(x)\}.$ 
\end{enumerate}
\end{enumerate}
\end{document}