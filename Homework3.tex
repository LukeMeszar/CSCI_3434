\documentclass{article}
\usepackage[english]{babel}
\usepackage{amsmath}
\usepackage{amssymb}
\usepackage{amsthm}
\usepackage{amsfonts}
\usepackage{blindtext}
\usepackage{systeme}
\usepackage{relsize}
\usepackage{enumerate}
\usepackage{commath}
\usepackage{mathrsfs}
\usepackage{bm}
\usepackage{float}
\usepackage{graphicx}
\usepackage{wrapfig}
\usepackage{mathtools}
\usepackage{subcaption}
\usepackage{tikz-cd}
\usetikzlibrary{automata,positioning}
%\usepackage{hyperref}
%\hypersetup{
% colorlinks=false,% hyperlinks will be black
%linkbordercolor=black,% hyperlink borders will be red
%pdfborderstyle={/S/U/W 0.25}% border style will be underline of width 1pt
%}
\usepackage{chngcntr}
\usepackage{MnSymbol}
\usepackage{tikz}
\usepackage[margin=0.75in]{geometry}
\usepackage{cleveref}

\newcommand{\C}{\mathbf{C}}
\newcommand{\R}{\mathbf{R}}
\newcommand{\Q}{\mathbf{Q}}
\newcommand{\Z}{\mathbf{Z}}
\newcommand{\N}{\mathbf{N}}
\newcommand{\E}{\mathbf{E}}
\newcommand{\B}{\mathbf{B}}
\newcommand{\F}{\mathbf{F}}
\newcommand{\U}{\mathbf{U}}
\newcommand{\V}{\mathbf{V}}
\newcommand{\J}{\mathbf{J}}
\newcommand{\x}{\mathbf{x}}
\newcommand{\y}{\mathbf{y}}
\newcommand{\z}{\mathbf{z}}
\newcommand{\Sb}{\mathbf{S}}
\newcommand{\Cs}{\mathscr{C}}
\newcommand{\As}{\mathscr{A}}
\newcommand{\I}{\textnormal{\textbf{I}}}
\newcommand{\Id}{\dot{\textnormal{\textbf{I}}}}
\newcommand{\Top}{\textnormal{\textbf{Top}}}
\newcommand{\hTop}{\textnormal{\textbf{hTop}}}
\newcommand{\Groups}{\textnormal{\textbf{Groups}}}
\newcommand{\Set}{\textnormal{\textbf{Set}}}
\newcommand{\Xib}{\mathbf{\Xi}}
\newcommand{\Sym}{\text{Sym}}
\newcommand{\prid}[1]{\langle #1 \rangle}
\newcommand{\apoly}[1]{a_{#1} + a_{#1}x+ a_{#1}x^2 + \cdots + a_{#1}x^n}
\newcommand{\dist}{\textnormal{dist}}
\newcommand{\ti}[1]{\textit{#1}}
\newcommand{\tb}[1]{\textnormal{\textbf{#1}}}
\newcommand{\es}{\varnothing}
\newcommand{\sst}{\subset}
\newcommand{\ssteq}{\subseteq}
\newcommand{\func}[3]{#1: #2 \to #3}
\newcommand{\inte}[1]{\textnormal{int}(#1)}
\newcommand{\bdr}[1]{\textnormal{bdry}(#1)}
\newcommand{\ifff}{if and only if }
\newcommand{\st}{such that }
\newcommand{\wrt}{with respect to }
\newcommand{\tspace}[1]{\text{T}_#1}
\newcommand{\mathdash}{\hbox{-}}
\newcommand{\diam}[1]{\textnormal{diam}(#1)}
\newcommand{\setst}{\hspace{1mm} | \hspace{1mm} }
\newcommand{\supp}{\textnormal{support}}
\newcommand{\clos}{\textnormal{closure}}
\newcommand{\rel}{\textnormal{rel }}
\newcommand{\Hom}{\textnormal{Hom}}
\newcommand{\obj}{\textnormal{obj}}
\newcommand{\varlisto}[2]{#1_1,#1_2,\ldots,#1_{#2}}
\newcommand{\varlistz}[2]{#1_0,#1_1,\ldots,#1_{#2}}
\newcommand{\finv}[2]{#1^{-1}(#2)}
\newcommand{\disu}{\rotatebox[origin=c]{90}{$\models$}}
\newcommand{\rank}{\textnormal{rank }}
\newcommand{\card}{\textnormal{card }}
\newcommand{\im}{\textnormal{im }}
\newcommand{\cls}{\textnormal{cls }}
\newcommand{\rev}{\textnormal{rev }}
\newcommand{\defeq}{\mathrel{\stackrel{\makebox[0pt]{\mbox{\normalfont\tiny def}}}{=}}}



%\renewcommand{\phi}{\varphi}
\renewcommand{\epsilon}{\varepsilon}


\newtheorem{theorem}{Theorem}[section]
\newtheorem{corollary}[theorem]{Corollary}
\newtheorem{proposition}{Proposition}[theorem]
\newtheorem{lemma}[theorem]{Lemma}
\theoremstyle{definition}
\newtheorem*{definition}{Definition}
\newtheorem*{solution*}{Solution}
\newtheorem*{remark}{Remark}
\newtheorem*{remarks}{Remarks}


\renewcommand\qedsymbol{$\blacksquare$}
\setcounter{section}{-1}

\counterwithin*{equation}{section}

\title{CSCI 3434 Theory of Computation Homework 3}
\author{Luke Meszar (Worked with Jeff Lipnick)}
\date{October 18, 2017}
\begin{document}
\maketitle
\begin{enumerate}
	\item[HW 5.2]
	Prove that the CFG 
	\[S \rightarrow aSb \setst bSa \setst SS \setst \epsilon\]
	generates the set of all strings over $\{a,b\}$ with equally many $a$'s and $b$'s. (Hint: Characterize elements of the set in terms of the graph of
	the function $\#b(y)-\#a(y)$ as y ranges over prefixes of $x$, as we did
	in Lecture 20 with balanced parentheses.)
	\begin{proof}
		Let $(*)$ denote the condition that $\#a(x) = \#b(x)$. Then a string is in this language \ifff it satisfies $(*)$. For the induction, we also need to assume that any $\alpha \in (N \cup \Sigma)^*$ also satisfies $(*)$. 
		
		For the base case, let $S \xrightarrow[G]{0} \alpha$. Then $\alpha = S$ and $(*)$ is satisfied since the number of $a'$s and $b$'s is 0. For the induction, suppose that $S \xrightarrow[G]{n+1} \alpha$. Let $\beta$ be the sentential form immediately proceeding $\alpha$. That is
		\[S \xrightarrow[G]{n} \beta \xrightarrow[G]{1} \alpha.\]
		Our induction hypothesis is that $\beta$ satisfied $(*)$. Now, the grammar can produce $\alpha$ from $\beta$ in four ways according to the production rules. First, we know that $\beta$ has the form $\beta_1S\beta_2$ where $\beta_1,\beta_2 \in (N \cup \Sigma)*$. We only need to include one $S$ since the others can be absorbed into $\beta_1$ or $\beta_2$. 
		
		If we apply the rule $S \to SS$ to $\beta$, we get $\alpha = \beta_1SS\beta_2$. Clearly this satisfied $(*)$ since $\beta$ satisfies $(*)$ and we haven't added any $a$'s or $b$'s. 
		
		If we apply the rule $S \to \epsilon$ to $\beta$, we get $\alpha = \beta_1\beta_2$. Clearly this satisfied $(*)$ since $\beta$ satisfies $(*)$ and we haven't added any $a$'s or $b$'s. 
		
		If we apply the rule $S \to aSb$ to $\beta$, we get $\alpha = \beta_1aSb\beta_2$. In this case, $\#a(\alpha) = \#a(\beta) + 1 = \#b(\beta) + 1 = \#b(\alpha)$ since $\#a(\beta) = \#b(\beta).$
		
		If we apply the rule $S \to bSa$ to $\beta$, we get $\alpha = \beta_1bSa\beta_2$. In this case, $\#a(\alpha) = \#a(\beta) + 1 = \#b(\beta) + 1 = \#b(\alpha)$ since $\#a(\beta) = \#b(\beta).$
		
		Thus, by induction, we have shown that if $S \xrightarrow[G]{*} x$, then $x$ has an equal number of $a$'s and $b$'s. 
		
		Now, we have to show that if $x$ satisfies $(*)$, then $S \xrightarrow[G]{*} x$. We will induct on $|x|$. Assume that $x$ satisfies $(*)$. 
		
		For the base case, if $|x| = 0,$ then $x = \epsilon$ and can be derived using the production $S \to \epsilon$. Now, assume $|x| >0$, then there are two cases. 
		
		Case 1: There exists a proper prefix (i.e $0 < |y| < |x|$) that satisfies $(*)$.  Then $x = yz$ for some $z$ where $0 < |z| < |x|$. We know that $z$ satisfies $(*)$ as well since
		\[\#a(z) = \#a(x)-\#a(y) = \#b(x)-\#b(y)=\#b(z).\]
		Then by our induction hypothesis, we know that $S \xrightarrow[G]{*} y$ and $S \xrightarrow[G]{*} y$. Then $x$ can be derived in the following way:
		\[S \xrightarrow[G]{1} SS \xrightarrow[G]{*} yS \xrightarrow[G]{*} yz = x.\]
		
		Case 2: There is no proper prefix of $x$ that satisfies $(*)$. Then, either $x = azb$ or $x = bza$. We can see that $z$ satisfies $(*)$ since
		\[\#a(z) = \#a(x)-1 = \#b(x)-1=\#b(z).\]
		So, by the induction hypothesis, $S \xrightarrow[G]{*} z$. Then, if $x = azb$, we can derive $x$ in the following way:
		\[S \xrightarrow[G]{1} aSb \xrightarrow[G]{*} azb = x.\]
		 If $x = bza$, we can derive $x$ in the following way:
		\[S \xrightarrow[G]{1} bSa \xrightarrow[G]{*} bza = x.\]
		In both cases, we have shown that if $x$ satisfies $(*)$, then $S \xrightarrow[G]{*} x$. 
	\end{proof}
	\item[HW 6.1] Prove that the following CFG $G$ in Greibach normal form generates
	exactly the set of nonnull strings over $\{a, b\}$ with equally many $a$'s and
	$b$'s:
	\begin{align*}
	S &\rightarrow aB \setst bA, \\
	A &\rightarrow aS \setst bAA \setst a, \\
	B &\rightarrow bS \setst aBB \setst b.
	\end{align*}
	(Hint: Strengthen your induction hypothesis to describe the sets of
	strings generated by the nonterminals $A$ and $B$: for $x \neq \epsilon$,
	\begin{align*}
	S &\xrightarrow[G]{*} x \iff \#a(x) = \#b(x),\\
	A &\xrightarrow[G]{*}\ x ???,\\
	B &\xrightarrow[G]{*}\ x ???.)\\
	\end{align*}
	\item[Misc. 1]
	Closure properties of CFLs. For each of the following, let $L_1$ and $L_2$ be arbitrary CFLs and determine whether the given language must be a CFL (YES) or can you find a counterexample (NO), i.e., particular choices of $L_1$ and $L_2$ such that the language is not a CFL. For the YES's give a short justification involving the construction of a grammar/PDA from the given grammars/PDAs. For t he NO's, use the CFL pumping lemma or known non-context-free languages to establish your counterexample.
	\begin{enumerate}
		\item $L_1L_2$
		\item $L_1 \cap L_2$
		\item $L_1 \cup L_2$ 
		\item $L_1^*$
	\end{enumerate} 
\begin{proof}\footnote{Referenced pg 195-196 in Kozen}
	\leavevmode
	\begin{enumerate}
		\item $L_1L_2$ 
		The concatenation of two context-free languages is context-free. Assume that $L_1$ is generated by $G_1 = (N_1,\Sigma_1,P_1,S_1)$ and $L_2$ is generated by $G_2 = (N_2,\Sigma_2,P_2,S_2)$. There may be the case where there are products in $P_1$ and $P_2$ that have the same name. To avoid any issues, rename any names that collide in $P_2$ to a new unique symbol that is not already being used in either set of productions. Denote the renamed productions for $G_2$ as $P_2'$. Consider the following grammar that accepts $L_1L_2$: $G = (N_1 \cup N_2, \Sigma_1 \cup \Sigma_2, P_1 \cup P_2' \cup \{S\}, S)$ where $S \rightarrow S_1S_2$. This grammar accepts the concatenation of $L_1$ and $L_2$ since it begins by producing any string in $L_1$ through expanding from $S_1$ and then it then producing any string in $L_2$ by expanding $S_2$. 
		\item $L_1 \cap L_2$
		Consider the two context-free languages:
		\[\{a^mb^mc^n \setst m,n \geq 0\}\]
		and
		\[\{a^mb^nc^n\setst m,n \geq 0\}.\]
		
		Then, the intersection of these two sets is 
		\[L = \{a^nb^nc^n \setst n \geq 0\}.\] 
		We can see this set is not regular by using the pumping lemma. Given $k \geq 0$, choose $z = a^kb^kc^k$. Then, for any $u,v,w,x,y$ such that $uvwxy = z$, $|vxy| \leq k$ and $|v| + |x| > 0$, we can break the problem down into a few cases. 
		
		Case 1. $vwx$ is all $a$'s which means $|vxy| \leq k$ since there are at most $k$ $a$'s. Then, after choosing $i = 2$, we find that $\#a(uv^2wx^2y) > \#b(z),\#c(z)$ so $uv^2wx^2y \not \in L$. There are practically identical cases if $vxy$ is all $b$'s or all $c$'s.
		
		Case 2. $vxy$ has both $a$'s and $b$'s. There is no way for $vxy$ to have any $c$'s since that would mean $|vxy| > k$ since $\#b(z) = k$. In particular, $v$ must contain $a$'s and $x$ must contain $b$'s. If either $v$ or $x$ is empty, then we degenerate to Case 1. Then, if we choose $i = 2$, we have two cases.First, $\#a(uv^2wx^2y) > \#c(z)$ and $\#b(uv^2wx^2y) > \#c(z)$. Second, it is also a possibility that if $v$ has both $a$'s and $b$'s, then after pumping, there string will $a$'s followed by $b$'s followed by $a$'s again which also breaks the language. A similar thing happens if $x$ contains both $a$'s and $b$'s. There is a practically identical case if $vxy$ has both $b$'s and $c$'s.
		
		Thus, $L$ is not context-free so CFL's are not closed under intersection. 
		\item The union of two context-free languages is context-free. Assume that $L_1$ is generated by $G_1 = (N_1,\Sigma_1,P_1,S_1)$ and $L_2$ is generated by $G_2 = (N_2,\Sigma_2,P_2,S_2)$. There may be the case where there are products in $P_1$ and $P_2$ that have the same name. To avoid any issues, rename any names that collide in $P_2$ to a new unique symbol that is not already being used in either set of productions. Denote the renamed productions for $G_2$ as $P_2'$. To accept $L_1 \cup L_2$, construct the following grammar: $G = (N_1 \cup N_2, \Sigma_1 \cup \Sigma_2, P_1 \cup P_2' \cup \{S\},S)$ where $S \rightarrow S_1 | S_2$. Essentially, this construction allows the production to move to either production rules that create strings in $L_1$ or strings in $L_2$. Once the production has moved to $S_1$ or $S_2$, the production proceeds in the same manner as in $G_1$ or $G_2$ respectively modulo the potential name changes in $P_2$.
		\item The $L_1^*$ is a context-free language. Assume that $L_1$ is generated by $G_1 = (N,\Sigma,P,S)$. Consider the following grammar that accepts $L_1^*$: $G = (N, \Sigma, P \cup \{S'\}, S')$ where $S' \rightarrow SS'|\epsilon$. This grammar allows any number of strings in $L_1$ concatenated together since the nonterminal $S' \rightarrow SS'$ can be expanded to $\underbrace{SS\ldots S}_{n \text{ times}}S'$ for $n \geq 0$. The $\epsilon$ term allows this expansion to end eventually so that there is no infinite recursion. 
	\end{enumerate}
\end{proof}
\item[Misc. 2] For each language, decide whether it is regular, not regular but context-free, or not context-free. Justify your answer. 
\begin{enumerate}
	\item $L_1 = \{a^k b^l a^m b^n \setst k = m \text{ or } l = n \}$ 
	\item $L_2 = \{a^k b^l a^m b^n \setst k = m \text{ and } l = n\}$
\end{enumerate}
\begin{solution*}
	\leavevmode
	\begin{enumerate}
			\item This language is not regular but it is context free. To show it is not regular, we will use the Pumping Lemma. Given $k \geq 0$, let $x = a^kb^k, y = a^k$ and $z = b^{k+1}$. Then, for any $u,v,w$ such that $y = uvw$ and $|v| \geq 0$, then choose $i = 2$. Then $\#a(uv^iw) > \#a(uvw)$. Then $xuv^iwz = a^kb^ka^lb^{k-1}$ where $l > k$. This is not in $L_1$ since neither the number of $a$'s are equal or the number of $b$'s. Thus $L_1$ is not regular. 
		
			To show this grammar is context-free, we will take the union of two grammars that together will generate $L_1$. Let $G_1$ be the following grammar:
			\begin{align*}
			S_1 &\rightarrow aTaB | B | \epsilon \\
			T &\rightarrow aTa | B \\
			B &\rightarrow B | b.
			\end{align*}
			This grammar generates strings of the form $\{a^k b^l a^m b^n \setst k = m\}$. This can be seen because anytime the grammar produces an $a$ on the left, it also produces one on the right. Since both rules that produce an $a$ are of the form $aTa$, we are guaranteed to have this invariant. The number of $b$'s it can produce on the right is in no way tied to the number of $b$'s produces on the right. This grammar can also produce a string with no $a$'s but some $b$'s and it can also produce the empty string.
			
			Let $G_2$ be the following grammar:
			\begin{align*}
			S_2 &\rightarrow AbRb | A | \epsilon \\
			T &\rightarrow bRb | A \\
			B &\rightarrow A | a.
			\end{align*}
			This grammar works almost identically to $G_1$ expect that it instead produces the language $\{a^k b^l a^m b^n \setst l = n\}$. Then, the following grammar produces: $L_1$:
			\begin{align*}
			S &\rightarrow S_1 | S_2 \\
			S_1 &\rightarrow aTaB | B | \epsilon \\
			T &\rightarrow aTa | B \\
			B &\rightarrow B | b \\
			S_2 &\rightarrow AbRb | A | \epsilon \\
			T &\rightarrow bRb | A \\
			B &\rightarrow A | a.
		\end{align*}
		\item This language is not context-free. We will prove this using the Pumping Lemma. Given $k \geq 0$, let $z = a^kb^ka^kb^k$. Let $uvwxy = z$ such that $|vwx| \leq k$ and $|v| + |x| > 0$. We will break this problem down into a few cases.
		
		Case 1: $vwx$ consists of all $a$'s which can happen in two parts of the string. Choose $i = 2$. Then $|v^iwx^i| > |vwx|$ which means $uv^iwx^iy$ is either of the form $a^lb^ka^kb^k$ or $a^kb^ka^lb^k$ where $l > k$. In either case, this string is not in $L_2$. There is a practically identical case when $vwx$ is all $b$'s. 
		
		Case 2: $vwx$ is of the form $a^mb^n$ where $m + n \leq k$. There is no possible way for a string of the form $a^mb^na^p$ since there are $k$ $b$'s in a row. In particular, $v$ must contain $a$'s and $x$ must contain $b$'s. If either $v$ or $x$ is empty, then we degenerate to Case 1. Then, if we choose $i \geq 2$, we get three cases. First, we get $uv^2wx^2y =  a^{k+s}b^{k+t}a^kb^k$ where $s+t > 0$ which is not in $L_2$. Second, we could get $a^kb^ka^{k+s}b^{k+s}$ where $s + t > 0$. Finally, it is also a possibility that if $v$ has both $a$'s and $b$'s, then after pumping, will be too many switches between $a$'s and $b$'s (more than 3) so the string will not be in the language. A similar thing can happen if $x$ contains $a$'s and $b$'s as well. 
		
		There is a practically similar case when $vwx$ is of the form $b^ma^n$ where $m + n \leq k$. We have shown that $L_2$ is not context-free.
	\end{enumerate}
\end{solution*}
\item[Misc. 3] Give a PDA accepting the following languages.  No proof required.
\begin{enumerate}
\item  $\{a^n b^m c^k \setst k = n + m\}$ 
\item $\{x \in \{a,b\}^* \setst \#a(x) = \#b(x)\}.$ 
\end{enumerate}
\begin{solution*}
	In both of these problems, I will use the CFG to PDA construction. This means that each language has to have a corresponding grammar in GNF. 
	\begin{enumerate}
		\item From class, we know that the following grammar produces this language:
		\begin{align*}
		S &\rightarrow aSC | bTC | aC | bC \\
		T &\rightarrow bTC | bC \\
		C & \rightarrow c.
		\end{align*}
		Then, using the CGF to PDA conversion algorithm, we get the following PDA:
		\[M \left(\{q\},\{a,b,c\},\{S,T,C\},\delta,q,S,\es\right)\]
		where 
		\begin{multline*}
		\delta = \{((q,a,S),(q,SC)),((q,b,S),(q,TC)),((q,a,S),(q,C)),\\((q,b,S),(q,C)),((q,b,T),(q,TC)),((q,b,T),(q,C)),((q,c,C),(q,\epsilon))\}
		\end{multline*}
		\item From 6.1, we know that the GNF grammar that generates this language is:
		\begin{align*}
		S &\rightarrow aB \setst bA, \\
		A &\rightarrow aS \setst bAA \setst a, \\
		B &\rightarrow bS \setst aBB \setst b.
		\end{align*}
		Then, using the CGF to PDA conversion algorithm, we get the following PDA:
		\[M = \left(\{q\},\{a,b\}, \{S,A,B\},\delta, q, S, \es\right)\]
		where 
		\begin{multline*}
		\delta = \{((q,a,S),(q,B)),((q,b,S),(q,A)),((q,a,A),(q,S)),\\((q,b,A),(q,AA)),((q,a,A),(q,\epsilon)),((q,b,B),(q,S)),((q,a,B),(q,BB)),((q,b,B),(q,\epsilon))\}
		\end{multline*}
	\end{enumerate}
\end{solution*}
\end{enumerate}
\end{document}